\documentclass[12pt]{article}

\usepackage{mystyle}

\title{Objectives for Information Extraction}
\author{
Justin T. Chiu
}
\date{\today}


\begin{document}
\maketitle

\begin{abstract}
In order to train a probabilistic information extraction without any
supervision at the level of text, we specify a generative model of
the relational data table and the text.
We choose a generative model.
\end{abstract}

\section{Introduction}
This is time for all good men to come to the aid of their party!

\section{Notation}
Our information extraction model is denoted $q(\bv \mid \by)$.

\paragraph{Outline}
The remainder of this article is organized as follows.
Section~\ref{previous work} gives account of previous work.
Our new and exciting results are described in Section~\ref{results}.
Finally, Section~\ref{conclusions} gives the conclusions.

\section{Previous work}\label{previous work}

\section{Results}\label{results}
In this section we describe the results.

\section{Conclusions}\label{conclusions}
We worked hard, and achieved very little.

%\bibliographystyle{abbrv}
%\bibliography{bib}

\end{document}
