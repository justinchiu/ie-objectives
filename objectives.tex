\documentclass[12pt]{article}

\usepackage{mystyle}

\title{Objectives for Information Extraction}
\author{
Justin T. Chiu
}
\date{\today}


\begin{document}
\maketitle

\begin{abstract}
Information extraction systems have historically operated above the word level.
Systems predict the relationship between an entity and value given
their locations in the text.
This requires observed locations of mentions, which requires annotations at the word level.
Any form of annotation at the word level does not scale as the size of the text
and the number of labels increases, and even more so if there is ambiguity.
In order to train a probabilistic information extraction without any
supervision at the level of text, we specify a model 
that, for each word, either chooses a triple from a knowledge base to explain
or chooses to explain nothing.
\end{abstract}

\section{Introduction}


\section{Notation}
Our information extraction model is denoted $q(\bv \mid \by)$.

\section{Two Perspectives}


\end{document}
