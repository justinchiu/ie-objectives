\documentclass[12pt]{article}

\usepackage{mystyle}

\title{Objectives for Information Extraction}
\author{
Justin T. Chiu
}
\date{\today}


\begin{document}
\maketitle

\begin{abstract}
Many recent information extraction systems predict the relationship between
an entity and value given the positions of their mentions in the text.
This requires requires words to be annotated as mentions.
Human annotation at the word level does not scale as the size of the text
and the number of labels increases, as annotators must read every word.
Automatic methods allow annotation to scale, but may introduce noise due to incorrect annotations.
In order to train a probabilistic information extraction model without mention
annotations, we specify a model that, for each word,
either chooses a triple from a knowledge base to explain or chooses to explain nothing.
\end{abstract}

\section{Problem Statement}

In relation extraction the goal is to extract facts from a passage of text.
Systems must convert facts expressed in natural language into a form
amenable to computation.
Facts consist of three components: entities, relation types, and values.
%(Example would be fastest to clear up what a relation type is)
The challenge is to not only extract facts from text, but also
justify the extractions by determining where those facts are mentioned.

A mention is a surface realization of an abstract object in text.
In relation extraction we justify extractions by identifying fact mentions.
As text is noisy, the realization of a fact may be difficult to locate.
We focus on locating fact mentions at the word level by identifying
individual words as value mentions, rather than entity or type mentions.
%(Why? Justify with generative model)

%Inspired by ,
We propose a model that first identifies value mentions at the word level,
aligns those mentions to an entity and relation type in order to obtain a fact, 
then aggregates word level decisions to resolve conflicts.
We assume access to a set of facts, henceforth referred to as a knowledge base (KB),
that is discussed in text.

\begin{comment}
Note on related work:
Except for \citet{zeng2018copy}, prior work has either assumed that the locations of
entities and values are given as input features or that the locations of entities and values
are observed at training time.
\end{comment}

The problem description is as follows:
given a text $x = x_0, \ldots, x_{I}$ we model the facts
$r = \set{(e_j, t_j, v_j)}_{j=0}^J$ expressed in that text
with respect to a schema that details all entities $e_j \in \mcE$,
relation types $t_j \in \mcT$, and all values $v_j \in \mcV$.
We assume that the schema of the KB $(\mcE, \mcT, \mcV)$ is known at all times,
and that the schema covers all facts of interest.
The set of facts $r$ is our knowledge base (KB),
and each individual fact $r_j$ is a record.
For brevity, let $e = \set{e_j}_{j=0}^J, t = \set{t_j}_{j=0}^J, v = \set{v_j}_{j=0}^J$
be the list of the entities, types, and values of the records in $r$ respectively.

We are primarily concerned with the scenario where we have an overcomplete KB schema with
respect to a specific passage of text.
This fits many scenarios in real world applications:
we may have thousands of entities of interest if our KB was pulled from an 
external source such as Freebase,
but the particular document we wish to analyze only discusses tens of entities,
only a few of which are present in our KB.

Given all entities $e$ and types $t$,
we reduce the construction of $r$ to predicting, for every $j\in 0,\ldots,J$,
the value $v_j$ corresponding to the entity $e_j$ and type $t_j$.
This reduction is inspired by the representation of a KB as a table:
the rows of the table are given by the entities $e_j \in \mcE$,
the columns by the types $t_j \in \mcT$,
and the cells of the table take on values $v_j \in \mcV$.
In the following section we propose a model for the distribution $p(v \mid x, e, t)$.

\section{Model}
We define a graphical model that performs extraction with justification. 
The model first extracts information at the word level,
then aggregates its choices for each word into an extraction at the sequence level.

The word level extraction process has three steps.
For each index $i \in 0, \ldots, I$ we perform
\begin{enumerate}
\item Value mention identification: Given a sequence of words $x$,
    we identify whether each word is a value mention with
    $p(m \mid x) = \prod_i p(m_i \mid x)$.
    Each $m_i \in \set{0,1}$.
    Not every word in a mention must be identified; it suffices to find
    at least one word in a value mention.
\item Alignment: Each value mention is then aligned to a 
    record in the knowledge base with $p(a \mid x,e,t) = \prod_i p(a_i \mid x,e,t)$,
    We align the word $x_i$ by choosing who (the entity)
    and what (the relation type) generate the possible value mention at index $i$.
    In particular, $a_i = j$ denotes the alignment to the record $r_{j}$
    with $a_i \in 0, \ldots, J$.
    We assume that each value mention aligns to a single record.
\item Translation: All value mentions are translated
    into a value from the KB schema with
    $p(z \mid x) = \prod_i p(z_i \mid x)$, with $z_i \in \mcV$.
\end{enumerate}
%We choose to locally normalize the word level distributions as our goal is to
%extract information from text, not condition on an existing KB.

Finally, we aggregate the word level information at the sequence level in order
to give a single distribution over the record values for $x$.
\begin{enumerate}
\setcounter{enumi}{3}
\item Aggregation $p(v \mid z,a,m) = \prod_j p(v_j \mid z,a,m)$:
    Given the word level values $z$, alignments $a$, value mention decisions $m$,
    we choose the sequence level value $v_j$.
\end{enumerate}

\begin{figure}[h]
\begin{center}
\resizebox {.3\columnwidth} {!} {
\begin{tikzpicture}
\node[obs] (Y) {$x$};
\node[latent, below=of Y](A) {$a_i$};
\node[latent, left=of A](C) {$m_i$};
\node[latent, right=of A](Vi) {$z_i$};

\node[obs, below=of C] (E) {$e_j$};
\node[obs, below=of A] (T) {$t_j$};
\node[obs, below=of Vi] (V) {$v_j$};

\plate {i} {(C)(A)(Vi)} {$I$};
\plate  {j} {(E)(T)(V)} {$J$};

\draw[->] (Y) -- (C);
\draw[->] (Y) -- (A);
\draw[->] (Y) -- (Vi);
%\draw[->] (A) -- (Vi);
\draw[->] (E) -- (A);
\draw[->] (T) -- (A);
%\draw[-] (Vi) -- (V);
%\draw[-] (A) -- (V);
%\draw[-] (C) -- (V);
%\draw[-] (E) to [bend right = 25] (V);
%\draw[-] (T) -- (V);
\draw[->] (Vi) -- (V);
\draw[->] (A) -- (V);
\draw[->] (C) -- (V);
\draw[->] (E) to [bend right = 25] (V);
\draw[->] (T) -- (V);

\end{tikzpicture}
} %% end resize
\end{center}
\caption{Our model predicts word-level values and alignments
then aggregates those choices over all indices $i$ to
predict values at the KB level.
}
\label{fig:infmodel}
\end{figure}

This gives us the following factorization of the relation extraction system:
\begin{equation}
\label{eqn:prob}
\begin{aligned}
p(v \mid x,e,t) &= \sum_{z,a,m} p(v,z,a,m\mid x,e,t)\\
&= \sum_{z,a,m} p(v\mid z,a,m,x,e,t) \prod_i p(z_i, a_i, m_i\mid x,e,t)\\
&= \sum_{z,a,m} \prod_j p(v_j\mid z,a,m,x,e,t) \prod_i p(z_i\mid x)p(a_i\mid x,e,t)p(m_i\mid x)\\
\end{aligned}
\end{equation}

\subsection{Parameterization}
Our model has four steps: mention identification, mention alignment, 
mention translation, and aggregation.
We parameterize the conditional distributions of each step below.

Let $\bh_i \in \R^d$ be a contextual embedding of the word $x_i$,
and $E$ an embedding function that maps entities and types
to vectors in $\R^{d'}$.
\begin{enumerate}
\item Identification: We use the contextual embedding to directly predict
whether a word is part of a value mention.
$$p(m_i \mid x) \propto \exp(W_m\bh_i), W_m \in \R^{2 \times d}$$
\item Alignment: We decompose the alignment distribution into a distribution over
entities $p(\epsilon_i \mid x,e)$ and types $p(\tau_i \mid x,t)$.
\begin{align*}
p(a_i \mid x,e,t) &= p(\epsilon_i \mid x,e)p(\tau_i \mid x,t)\\
p(\epsilon_i \mid x) &\propto \exp(E(e_{\epsilon_i})^T W_e \bh_i)\\
p(\tau_i \mid x) &\propto \exp(E(\tau_{a_i})^T W_t \bh_i)
\end{align*}
with $W_e \in \R^{d' \times d},W_t \in \R^{d' \times d}$.
\item Translation: We use the contextual embedding to translate a word
into a value.
$$p(z_i \mid x) \propto \exp(W_z \bh_i), W_z \in \R^{|\mcV| \times d}$$
\item Aggregation:
If there exists an index that is a mention that is aligned to $r_j$
we allow it to vote on the value $v_j$, otherwise we ignore the text
and use a prior distribution over values
$p(v_j \mid e_j, t_j) \propto \exp(E(v_j)^TW_v [E(e_j),E(t_j)])$.
\begin{align*}
p(v_j \mid z,a,m,e,t) &\propto \begin{cases}
    \prod \exp(\psi(v_j, z_i,a_i,m_i,e,t)),  & \exists i, m_i = 1 \wedge a_i = j\\
    %p(v_j \mid e_j,t_j), & \textrm{otherwise}
    \exp(E(v_j)^TW_v [E(e_j),E(t_j)]), & \textrm{otherwise}
\end{cases}\\
\psi(v_j, z_i, a_i, m_i,e,t) &= 1(v_j = z_i, a_i = j, m_i=1)%\\
%p(v_j \mid e,t) & \propto \exp(E(v_j)^TW_v [E(e),E(t)])
\end{align*}
\end{enumerate}

\section{Training and Inference}
To train a latent variable model, we must marginalize over the unobserved RVs
and maximize the likelihood of the observed.
Ideally, we would optimize the following objective
\begin{equation}
\log p(r \mid x) = \log \sum_{z,a,m} p(r,z,a,m \mid x)
\end{equation}
However, maximizing $\log p(v \mid x)$ directly is very expensive for this model
as the summation over $z,a,m$ is intractable.
The summation over $z,a,m$ has computational complexity $O((|\mcV|\cdot J\cdot 2)^{I})$,
which is exponential in the length of the text.
Additionally, the size of the KB $J$ may be large as well.

We therefore resort to approximate inference,
specifically amortized variational inference.

\subsection{Inference network}
Our first approach is to introduce an inference network $q(z,a,m\mid v,x,e,t)$
and optimize the following lower bound on the marginal likelihood
with respect to the parameters of both $p$ and $q$:
\begin{equation}
\label{eqn:lowerbound}
\log p(v\mid x) \geq
\Es{q(z,a,m\mid v,x,e,t)}{\log \frac{p(v,z,a,m\mid x,e,t)}{q(z,a,m\mid v,x,e,t)}}
\end{equation}

We propose to parameterize $q(z,a,m\mid v,x,e,t)$ as follows.
We decompose 
\begin{equation}
\begin{aligned}
q(z,a,m\mid v,x,e,t) &= q(z \mid a,v,x)q(a\mid v,x,e,t)q(m \mid v,x)\\
&= \prod_i q(z_i \mid a,v,x)q(a_i \mid v,x,e,t)q(m_i \mid v,x)
\end{aligned}
\end{equation}
The conditional distributions of our inference network
are very similar to the relation extraction model,
but they condition on the values $v$.

Let $\bh_i \in \R^d$ be a contextual embedding of the word $x_i$.
We use attention weights over records to get a weighted representation
of the records of the KB for each index $i$:
\begin{align*}
\bg_{r_j} &= [E(e_j), E(t_j), E(v_j)]\\
\alpha_j &\propto \exp(\bg_{r_j}^T W_\alpha \bh_i)
\end{align*}
The inference network is given by
\begin{enumerate}
\item The value mention model $q(m_i \mid v,x)$ 
    has access to the values $v$ from the KB, which it conditions on
    when detecting value mentions. 
    $$p(m_i \mid v,x) = W_m' \textrm{MLP}([\sum_j \alpha_j \cdot \bg_{r_j}, \bh_i]), W'_m \in \R^{2 \times d}$$
\item The alignment model $q(a_i \mid v,x,e,t)$
    uses a contextual representation of each $x_i$ and chooses a record.
    In contrast to $p(a\mid x,e,t)$, this model has access to values as well.
    We use the attention weights to parameterize the distribution
    $p(a_i \mid x) = \alpha_{a_i}$.
\item The translation model $q(z_i \mid a,v,x) = 1(z_i = v_{a_i})$
    conditions on the alignments $a$ and ensures the chosen $z$ is consistent
    with the alignments. 
\end{enumerate}

One concern is that the model may learn to never rely on the text for extraction,
setting $m_i = 0$ at every index.
We can avoid this by initializing $q(z)$ to ensure that for words $x \in \mcV$ 
we have $q(z = x)$ is high, biasing the translation model towards transliteration
at the start of training.

\subsection{Approximate the posterior of a generative model}
Alternatively, we may decompose the training of our extraction system $p(v\mid x)$ into two stages:
In the first stage we train $p(z,a,m\mid x,e,t)$ to approximate the posterior
of a conditional model of text given a complete KB $q(x,z,a,m \mid e,t,v)$.
This has the benefit of allowing us to exert control over where value mentions are detected
through our design of the text model $q$.

In the second stage, we have two choices:
a) train $p(v \mid z,a,m,x,e,t)$ to approximate the posterior of a full generative model of text and
the values of KB $q(x,v\mid e,t)$.
b) train $p(v \mid z,a,m,x,e,t)$ using the following lower bound:
\begin{equation}
\label{eqn:lowerbound2}
\log p(v\mid x) \geq
\Es{p(z,a,m\mid x,e,t)}{\log p(v\mid z,a,m, x,e,t)}
\end{equation}
Ideally the bound in Eqn.~\ref{eqn:lowerbound2}
should not be looser than the one presented in Eqn.~\ref{eqn:lowerbound},
as conditioning on the observed values of a KB should not reduce the entropy of
a good alignment model.

How do we split the gradient over time?

\section{Evaluation}
Although we have a model over the values of all records,
evaluation does not include the final distribution over all record values.
As we assumed that the KB contained a superset of the facts contained in
a sequence of text, we are evaluate whether the model can discover the subset of facts that 
are expressed in the text.
We therefore perform extraction by using the marginal distributions
$q(z),q(a),q(c)$ to value mentions as well as entities and types,
giving us facts.

\bibliography{bib}
\bibliographystyle{acl_natbib}

\end{document}
