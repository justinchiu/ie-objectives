\documentclass[12pt]{article}

\usepackage{mystyle}

\title{Relation Extraction}
\author{
Justin T. Chiu
}
\date{\today}


\begin{document}
\maketitle

\begin{abstract}
Many recent relation extraction systems predict the relationship between
an entity and value given the positions of their mentions in the text.
This requires requires words to be annotated as mentions.
Human annotation at the word level does not scale as the size of the text
and the number of labels increases, as annotators must read every word.
Automatic annotation methods allow the annotation process to scale,
but may introduce noise due to incorrect annotations.
In order to train a probabilistic information extraction model without mention
annotations, we specify a model that, for each word,
either chooses a triple from a knowledge base to explain or chooses to explain nothing.
\end{abstract}

\section{Problem Statement}
\begin{comment}
NOTE on RE vs KBP: KBP focuses on modeling the full database, whereas
in RE the emphasis is on explaining extractions.
KBP usually has a more complicated model over the KB,
while RE sometimes makes simplifying assumptions and could make a poor
KBP system.
Example: if triple double appears in text and someone scored < 10 PTS,
then it's more likely they more likely had > 10 REB.
The difference between RE and KBP is subtle. Both fall under
information extraction, and the models may overlap in many ways.
In fact, latent variable model approaches to RE are usually able to perform KBP as well.
However, the focus is usually not on having good aggregate extraction metrics,
but rather about identifying where facts are in text.
This leads to simplifying assumptions in the model of the KB itself.
On the other hand, KBP moves the difficulty from explaining extractions to
faithfully modeling the KB.
Typically this results in a challenging inference task, as the structure
of the KB may be very complex.
\end{comment}

In relation extraction the goal is to extract facts from a passage of text.
Systems must convert facts expressed in natural language into a form
amenable to computation.
Facts consist of three components: entities, relation types, and values.
%(Example would be fastest to clear up what a relation type is)
The challenge is to not only extract facts from text, but also
justify the extractions by determining where those facts are mentioned.

A mention is a surface realization of an abstract object in text.
In relation extraction we justify extractions by identifying fact mentions.
As text is noisy, the realization of a fact may be difficult to locate.
We focus on locating fact mentions at the word level by identifying
individual words as value mentions, rather than entity or type mentions.
%(Why? Justify with generative model)

\begin{comment}
Note on related work:
Except for \citet{zeng2018copy}, prior work has either assumed that the locations of
entities and values are given as input features or that the locations of entities and values
are observed at training time.
\end{comment}

The problem description is as follows.
We focus on the domain of basketball summaries:
given a written summary of a basketball game $x = x_1, \ldots, x_{I}$
we model the aligned box score $\set{(e_j, t_j, v_j)}_{j=1}^J$
consisting of entities $e_j$,
relation types $t_j$, and all values $v_j \in \mcV$.
The set of facts is our knowledge base (KB).
Let $d = \set{(e_j, t_j)}_{j=1}^J$ and $v = \set{v_j}_{j=0}^J$.
The KB $(d,v)$ can be viewed as a data table where
$d$ defines a flattened representation of the rows and columns and
$v$ gives the values of the cells.
For example, we may have
$
d = \set{
    (e_1 = \textrm{John}, t_1 = \textrm{Points}),
    (e_2 = \textrm{John}, t_2 = \textrm{Rebounds}),
    \ldots
}
$
with 
$ v = \set{ v_1 = 19, v_2 = 12, \ldots }$ 
aligned to the brief summary $x = \textrm{John scored 19 points}$.
Our goal is to locate and extract facts from $x$.

\begin{comment}
(move somewhere else)
We are primarily concerned with the scenario where we have an overcomplete KB schema with
respect to a specific passage of text.
This fits many scenarios in real world applications:
we may have many entity and type pairs in our data table,
but a summary may discuss only a small salient subset of players and statistics.
\end{comment}

As our goal is to locate fact mentions, modeling just the KB $(d,v)$ given the text $x$
is not sufficient.
We therefore propose a model that first identifies words as value mentions,
aligns those mentions to an entity and relation type in order to obtain a fact, 
then aggregates word level decisions to resolve conflicts.
%We choose to identify value mentions as the location of a fact as entity mentions
%may be obscured through coreference and relation types may not be mentioned explicitly.

\section{Model}
We define a graphical model that performs extraction with justification. 
The model first extracts information at each word,
then aggregates its choices for each word into an extraction at the sequence level.

The word level extraction process has three steps.
For each index $i \in 1, \ldots, I$ we perform
\begin{enumerate}
\item Value mention identification: Given a sequence of words $x$,
    we identify whether each word is a value mention with
    $p(m \mid x) = \prod_i p(m_i \mid x)$.
    Each $m_i \in \set{0,1}$.
    Not every word in a mention must be identified; it suffices to find
    at least one word in a value mention.
\item Alignment: Each value mention is then aligned to a 
    record in the knowledge base with $p(a \mid x,d) = \prod_i p(a_i \mid x,d)$,
    We align the word $x_i$ by choosing who (the entity)
    and what (the relation type) generate the possible value mention at index $i$.
    In particular, $a_i = j$ denotes the alignment to the record $r_{j}$
    with $a_i \in 1, \ldots, J$.
    We assume that each value mention aligns to a single record.
\item Translation: All value mentions are translated
    into a value from the KB schema with
    $p(z \mid x) = \prod_i p(z_i \mid x)$, with $z_i \in \mcV$.
\end{enumerate}
%We choose to locally normalize the word level distributions as our goal is to
%extract information from text, not condition on an existing KB.

Finally, we aggregate the word level information at the sequence level in order
to give a single distribution over the record values for $x$.
\begin{enumerate}
\setcounter{enumi}{3}
\item Aggregation $p(v \mid z,a,m,d) = \prod_j p(v_j \mid z,a,m,d_j)$:
    Given the word level values $z$, alignments $a$, value mention decisions $m$,
    and data table $d$ we choose the sequence level value $v_j$.
\end{enumerate}

\begin{figure}[h]
\begin{center}
\resizebox {.35\columnwidth} {!} {
\begin{tikzpicture}
\node[obs] (Y) {$x$};
\node[latent, below=of Y](A) {$a_i$};
\node[latent, left=of A](C) {$m_i$};
\node[latent, right=of A](Vi) {$z_i$};

\node[obs, below=of A] (D) {$d_j$};
\node[obs, below=of Vi] (V) {$v_j$};

\plate {i} {(C)(A)(Vi)} {$I$};
\plate  {j} {(D)(V)} {$J$};

\draw[->] (Y) -- (C);
\draw[->] (Y) -- (A);
\draw[->] (Y) -- (Vi);
%\draw[->] (A) -- (Vi);
\draw[->] (D) -- (A);
%\draw[-] (Vi) -- (V);
%\draw[-] (A) -- (V);
%\draw[-] (C) -- (V);
%\draw[-] (E) to [bend right = 25] (V);
%\draw[-] (T) -- (V);
\draw[->] (Vi) -- (V);
\draw[->] (C) -- (V);
\draw[->] (A) -- (V);
\draw[->] (D) -- (V);

\end{tikzpicture}
} %% end resize
\end{center}
\caption{Our model predicts word-level values and alignments
then aggregates those choices over all indices $i$ to
make a single decision for each value.
Each word has the following latent variables:
the mention $m_i \in \set{0,1}$ indicates whether word $x_i$ is a value mention,
the alignment $a_i$ gives what fact $x_i$ aligns to,
and the value $z_i$ gives the canonical value that $x_i$ translates to.
}
\label{fig:infmodel}
\end{figure}

This gives us the following factorization of the relation extraction system:
\begin{equation}
\label{eqn:prob}
\begin{aligned}
p(v \mid x,d) &= \sum_{z,a,m} p(v,z,a,m\mid x,d)\\
&= \sum_{z,a,m} p(v\mid z,a,m,x,d) p(z, a, m\mid x,d)\\
&= \sum_{z,a,m} \left(\prod_j p(v_j\mid z,a,m,x,d)\right)
\left(\prod_i p(z_i\mid x)p(a_i\mid x,d)p(m_i\mid x)\right)
\end{aligned}
\end{equation}

\subsection{Parameterization}
Our model has four steps: mention identification, mention alignment, 
mention translation, and aggregation.
We parameterize the conditional distributions of each step below.

Let $\bh_i \in \R^d$ be a contextual embedding of the word $x_i$,
and $E$ an embedding function that maps entities and types to vectors in $\R^{d'}$.
\begin{enumerate}
\item Identification: We use the contextual embedding to directly predict
whether a word is part of a value mention.
$$p(m_i \mid x) \propto \exp(W_m\bh_i), W_m \in \R^{2 \times d}$$
\item Alignment: We decompose the alignment distribution into a distribution over
entities $p(\epsilon_i \mid x,d)$ and types $p(\tau_i \mid x,d)$.
\begin{align*}
p(a_i \mid x,d) &= p(\epsilon_i= e_{a_i}\mid x,d)p(\tau_i = t_{a_i} \mid x,d)\\
p(\epsilon_i \mid x,d) &\propto \exp(E(e_{\epsilon_i})^T W_e \bh_i)\\
p(\tau_i \mid x,d) &\propto \exp(E(\tau_{a_i})^T W_t \bh_i)
\end{align*}
with $W_e \in \R^{d' \times d},W_t \in \R^{d' \times d}$.
\item Translation: We use the contextual embedding to translate a word
into a value.
$$p(z_i \mid x) \propto \exp(W_z \bh_i), W_z \in \R^{|\mcV| \times d}$$
\item Aggregation:
If there exists an index that is a mention and is also aligned to $r_j$
we allow it to vote on the value $v_j$, otherwise we ignore the text
and use a prior distribution over values
$p(v_j \mid d_j) \propto \exp(E(v_j)^TW_v [E(d_j)])$.
\begin{align*}
p(v_j \mid z,a,m,d) &\propto \begin{cases}
    \prod \exp(\psi(v_j, z_i,a_i,m_i,d)),  & \exists i, m_i = 1 \wedge a_i = j\\
    %p(v_j \mid e_j,t_j), & \textrm{otherwise}
    \exp(E(v_j)^TW_v [E(d_j)]), & \textrm{otherwise}
\end{cases}\\
\psi(v_j, z_i, a_i, m_i,d) &= 1(v_j = z_i, a_i = j, m_i=1)%\\
%p(v_j \mid e,t) & \propto \exp(E(v_j)^TW_v [E(e),E(t)])
\end{align*}
\end{enumerate}

\section{Training and Inference}
To train a latent variable model, we must marginalize over the unobserved RVs
and maximize the likelihood of the observed.
Ideally, we would optimize the following objective
\begin{equation}
\log p(v \mid x,d) = \log \sum_{z,a,m} p(v,z,a,m \mid x,d)
\end{equation}
However, maximizing $\log p(v \mid x,d)$ directly is very expensive for this model
as the summation over $z,a,m$ is intractable.
The summation over $z,a,m$ has computational complexity $O((|\mcV|\cdot J\cdot 2)^{I})$,
which is exponential in the length of the text.
Additionally, the size of the KB $J$ may be large as well.

We therefore resort to approximate inference,
specifically amortized variational inference.

\subsection{Inference network}
Our first approach is to introduce an inference network $q(z,a,m\mid v,x,d)$
and optimize the following lower bound on the marginal likelihood
with respect to the parameters of both $p$ and $q$:
\begin{equation}
\label{eqn:lowerbound}
\log p(v\mid x) \geq
\Es{q(z,a,m\mid v,x,d)}{\log \frac{p(v,z,a,m\mid x,d)}{q(z,a,m\mid v,x,d)}}
\end{equation}

We propose to parameterize $q(z,a,m\mid v,x,d)$ as follows.
We decompose 
\begin{equation}
\label{eqn:elbo}
\begin{aligned}
q(z,a,m\mid v,x,d) &= q(z \mid a,v,x)q(a\mid v,x,d)q(m \mid v,x)\\
&= \prod_i q(z_i \mid a,v,x)q(a_i \mid v,x,d)q(m_i \mid v,x)
\end{aligned}
\end{equation}
The conditional distributions of our inference network
are very similar to the relation extraction model,
but they condition on the values $v$.

Let $\bh_i \in \R^d$ be a contextual embedding of the word $x_i$.
We use attention weights over records to get a weighted representation
of the records of the KB for each index $i$:
\begin{align*}
\bg_{r_j} &= [E(d_j), E(v_j)]\\
\alpha_j &\propto \exp(\bg_{r_j}^T W_\alpha \bh_i)
\end{align*}
The inference network is given by
\begin{enumerate}
\item The value mention model $q(m_i \mid v,x)$ 
    has access to the values $v$ from the KB, which it conditions on
    when detecting value mentions. 
    $$p(m_i \mid v,x) = W_m' \textrm{MLP}([\sum_j \alpha_j \cdot \bg_{r_j}, \bh_i]), W'_m \in \R^{2 \times d}$$
\item The alignment model $q(a_i \mid v,x,d)$
    uses the attention weights to parameterize the alignment
    $p(a_i \mid x) = \alpha_{a_i}$.
\item The translation model $q(z_i \mid a,v,x) = 1(z_i = v_{a_i})$
    conditions on the alignments $a$ and ensures the chosen $z$ is consistent
    with the alignments. 
\end{enumerate}

We use the score function gradient estimator to perform gradient ascent on the objective
in Equation~\ref{eqn:elbo} with respect to both $p$ and $q$.
We utilize a leave-one-out baseline for variance reduction.

One concern is that the model may learn to never rely on the text for extraction,
setting $m_i = 0$ at every index.
We can avoid this by initializing $q(z)$ to ensure that for words $x \in \mcV$ 
we have $q(z = x)$ is high, biasing the translation model towards transliteration
at the start of training.

\section{Evaluation}
As we assumed that the KB contained a superset of the facts contained in
a sequence of text, we are interested in evaluating whether the model can discover
and locate the subset of facts that are expressed in the text.

We evaluate by determining whether the model can identify which facts in a KB
are expressed in text.
Given a ground truth set of facts extracted by a human annotator,
we compute the precision and recall of the extractions by the model.
We perform extraction by finding $\arg\max_{z_i,a_i,m_i}q(z_i),q(a_i),q(m_i)$
for each word $x_i$.


\begin{comment}

\subsection{Approximate the posterior of a generative model}
Alternatively, we may introduce a generative model of text $q(x, v)$ 
that inverts the relation extraction model $p(v \mid x)$ and optimize the following objective:
\begin{equation}
\log q(x,v) - KL[p(v,z,a,m \mid x) || q(v,z,a,m \mid x)]
\end{equation}
which entails training the generative model of text while approximating its
posterior with the information extraction system.

decompose the training of our extraction system $p(v\mid x)$ into two stages:
In the first stage we train $p(z,a,m\mid x,d)$ to approximate the posterior
of a conditional model of text given a complete KB $q(x,z,a,m \mid v,d)$.
This has the benefit of allowing us to exert control over where value mentions are detected
through our design of the text model $q$.

In the second stage, we have two choices:
a) train $p(v \mid z,a,m,x,d)$ to approximate the posterior of a full generative model of text and
the values of KB $q(x,v\mid d)$.
b) train $p(v \mid z,a,m,x,d)$ using the following lower bound:
\begin{equation}
\label{eqn:lowerbound2}
\log p(v\mid x,d) \geq
\Es{p(z,a,m\mid x,d)}{\log p(v\mid z,a,m, x,d)}
\end{equation}
Ideally the bound in Eqn.~\ref{eqn:lowerbound2}
should not be looser than the one presented in Eqn.~\ref{eqn:lowerbound},
as conditioning on the observed values of a KB should not reduce the entropy of
a good alignment model.

How do we split the gradient over time?

\end{comment}

\bibliography{bib}
\bibliographystyle{acl_natbib}

\end{document}
